 \subsection*{Introduction}

{\bfseries Starlight} est un petit jeu en deux dimensions se jouant sur une carte rectangulaire, comportant une source de lumière, émettant un rayon rectiligne.

Le but du jeu est d\textquotesingle{}atteindre une cible avec ledit rayon, en évitant les obstacles via notamment des miroirs réfléchissant la lumière.;

\subsection*{Rules}

Starlight est un puzzle à deux dimensions se jouant sur une carte rectangulaire. Le but du jeu est de dévier un rayon lumineux d\textquotesingle{}une source vers une cible en évitant certains obstacles. Plus particulièrement, on trouve les éléments suivants sur une carte \+:
\begin{DoxyItemize}
\item une unique source \+: cet élément émet un rayon lumineux d\textquotesingle{}une longueur d\textquotesingle{}onde donnée sous un certain angle,
\item une unique cible (ou destination ) \+: cet élément doit être éclairé par un rayon lumineux pour remporter la partie,
\item un ensemble de miroirs \+: un miroir est un objet réfléchissant la lumière d\textquotesingle{}un seul côté suivant le schéma naturel de la réflexion de la lumière, Plus particulièrement, un rayon incident à un miroir sous un angle  i sera réfléchi sous le même angle  r,
\item un ensemble de murs \+: les murs ne réfléchissent pas la lumière. Tout rayon incident à un mur ne se propage pas, et \char`\"{} s\textquotesingle{}arrête \char`\"{} donc là où il y est incident,
\item un ensemble de lentilles. Les lentilles sont des objets transparents qui ne laissent passer un rayon lumineux que dans un certain intervalle de longueur d\textquotesingle{}onde \mbox{[}m ,n\mbox{]}. Si un rayon lumineux possède une longueur d\textquotesingle{}onde ν telle que m $<$= ν $<$= n, il traverse la lentille sans subir aucune modification. Sinon, la lentille stoppe le rayon (elle se comporte comme un mur),
\item un ensemble de cristaux \+: un cristal est un élément transparent qui modifie la longueur d\textquotesingle{}onde d\textquotesingle{}un rayon, en l\textquotesingle{}augmentant ou la dimi-\/ nuant. Tout rayon qui traverse un cristal le traverse donc sans subir de modification de trajectoire, mais voit sa longueur d\textquotesingle{}onde modifiée,
\item un ensemble de bombes. Les bombes sont des objets qui, si éclairés, explosent et font automatiquement perdre la partie au joueur,
\item un ensemble de rayons initialement émis par la source du jeu. Ceux-\/ci sont rectilignes et se réfléchissent sur les miroirs. Un rayon est donc un segment de droite. Un rayon possède également une autre caractéristique \+: sa longueur d\textquotesingle{}onde. La longueur d\textquotesingle{}onde d\textquotesingle{}un rayon permet de déterminer, comme mentionné ci-\/dessus, si oui ou non un rayon traverse une lentille. Elle est modifiée par un cristal. 
\end{DoxyItemize}