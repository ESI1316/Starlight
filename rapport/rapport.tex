\documentclass[a4paper,11pt]{report}

\usepackage[utf8]{inputenc}
\usepackage[french]{babel}
\usepackage[T1]{fontenc}
\usepackage[pdftex]{graphicx} 
\usepackage{mathtools}
\usepackage{listings}
\usepackage{amsfonts}
\usepackage{tikz}
\usepackage{url} 
\usepackage[bookmarks, colorlinks=false, pdfborder={0 0 0}, pdftitle={Starlight
: rapport}, pdfauthor={Kriwin Paul, Placentino Simon}, pdfsubject={Starlight},
pdfkeywords={C++, ISO/IEC 14882:2011, Starlight, Projet}]{hyperref} 

\newcommand{\HRule}{\rule{\linewidth}{0.5mm}}

\begin{document}
\begin{titlepage}
	\begin{center}

		\includegraphics[width=0.15\textwidth]{./logo}\\[1cm]

		\textsc{\LARGE H.E.B. Ecole Superieur d'Informatique}\\[1.5cm]

		\textsc{\Large Laboratoire de C++ : projet 2}\\[0.5cm]

		\HRule \\[0.4cm]
		{\huge \bfseries Starlight \\[0.4cm]}
		\HRule \\[1.5cm]

		\noindent
		\begin{minipage}[t]{0.4\textwidth}
			\begin{flushleft} \large
				\emph{Auteurs:}\\
				Paul \textsc{Kriwin}\\
				Simon \textsc{Placentino}
			\end{flushleft}
		\end{minipage}%
		\begin{minipage}[t]{0.4\textwidth}
			\begin{flushright} \large
				\emph{Titulaire du cours:} \\
				Dr.~Romain \textsc{Absil}
			\end{flushright}
		\end{minipage}

		\vfill

		{\large \today}

	\end{center}
\end{titlepage}

\tableofcontents

\chapter{Introduction}



\chapter[Les classes]{Présentation succinte des classes}
\section[Les objets géométriques]{geometry}

\subsection[Ellipse]{ellipse.hpp}
\begin{center}
	\begin{tikzpicture}
		\draw[gray, very thin] (-2, -1) grid (2, 2);
		\draw (0,0) ellipse (2cm and 1cm);
	\end{tikzpicture} 
\end{center}
Une ellipse est un objet géométrique à deux dimensions représentée par
une courbe plane fermée obtenu par découpe d'un cône sur un plan. Si ce dernier
est perpendiculaire à l'axe du cône, l'ellipse sera alors un cercle.
Éléments caractéristiques d'une ellipse :
\begin{itemize}
	\item une coordonnée cartésienne de son centre,
	\item une distance séparant le centre de l'intersection avec une paralelle à
		l'axe des ordonnées tangante à l'ellipse voulue,
	\item une distance sépérant le centre de l'intersection avec une paralelle à
		l'axe des abscisse tangante à l'ellipe voulue.
\end{itemize}
Ces éléments nous permettront de tracer une ellipse selon cette équation :
$$\frac{(x - c_x)^2}{{x_{radius}}^2} + \frac{(y - c_y)^2}{{y_{radius}}^2} = 1$$

Cette classe peut tout à fait être instancié en objet géométrique elliptique
et possède des méthodes d'interractions avec une droite. 
\begin{lstlisting}[frame=single,language=C++]
Ellipse::getIntersectionsPoints(Line);
\end{lstlisting}
Dans le contexte présent, certains éléments du jeu seront des ellipse par le 
phénomène d'héritage mis en place dans le paradigme orienté objet de C++.

\subsection[Droite]{line.hpp}
\begin{center}
	\begin{tikzpicture}
		\draw[gray, very thin] (0,0) grid (3,3);
		\draw[dashed] (-1, -1) -- (0, 0);
		\draw (0,0) -- (3, 3);
		\draw[dashed] (3,3) -- (4, 4);
	\end{tikzpicture} 
\end{center}
Une droite est une ligne sans épaisseur, rectiligne et infinie dans le plan. 
Pour exister, une droite aura besoin :
\begin{itemize}
	\item d'un coefficiant angulaire $m = \frac{\Delta y}{\Delta x}$
		représentant la distance à parcourir sur l'axe des ordonnées pour une unité de
		distance sur l'axe des abcisses.
	\item d'un terme indépendant $p = \frac{y}{m \cdot x}$ représentant le décalage de chaque point
		sur l'axe des ordonnées,
	\item ou de deux points de coordonnées dans le plan,
	\item ou d'un point de coordonnées dans le plan et d'un coefficiant
		angulaire.
\end{itemize}
Ces éléments nous permettent de tracer une droite selon cette équaiton :
$$ y = m \cdot x + p $$

\subsection[Rectangle]{rectangle.hpp}
\begin{center}
	\begin{tikzpicture}
		\draw (1, 1) -- (4, 1);
		\draw (4, 1) -- (4, 2.5);
		\draw (4, 2.5) -- (1, 2.5);
		\draw (1, 2.5) -- (1, 1);
	\end{tikzpicture} 
\end{center}
Un rectangle est une forme géométrique à 4 segments de droite\footnote{Un
	segment de droite est une partie de droite délimitée par deux points non
confondus} 
parallèle deux à deux.Ceux-ci vont donc former 4 angles droit ($\frac{\pi}{2} rad$)
Cette forme peut être représentée par :
\begin{itemize}
	\item la coordonnée du coin supérieur gauche $Sg = (x, y)$
	\item la grandeur des deux segments formant un angle de $\frac{\pi}{2} rad$
		en ce point $hauteur$ et $largeur$.
\end{itemize}

Ainsi, il sera aisé de déterminer la position des autres coins 
\begin{itemize}
	\item $Sd = (Sg_x + largeur, Sg_y)$
	\item $Ig = (Sg_x , Sg_y + hauteur)$
	\item $Id = (Sg_x + largeur, Sg_y + hauteur)$
\end{itemize}

Il existe plusieurs cas particuliers de droite :
\begin{itemize}
	\item une droite parallèle à l'axe des ordonnées qui aura pour équation
		$$x = a, a \in \mathbb{R}, $$
	\item une droite parallèle à l'axe des abscisses qui aura pour équation
		$$y = b, b \in \mathbb{R}, $$
\end{itemize}
\subsection[Point]{point.hpp}
Un point est un objet mathématique permettant de situer un element dans un plan
ou dans l'espace. Dans notre cas, plus spécifiquement dans un plan à deux
dimensions. Celui-ci peut-être représenté de plusieurs manières dans le plan 
cartésien\footnote{\url{http://www.cslaval.qc.ca/sitsatlll/maths2003/cartesien.html}} : 
\begin{itemize}
	\item sous la forme d'une coordonnées cartésienne à l'aide de 
		\begin{itemize}
			\item une origine,
			\item deux vecteurs partant de cette origine et perpendiculaires,
				$$P = (\vec{x}, \vec{y})$$
		\end{itemize}
	\item et sous la forme d'une coordonnée polaire à l'aide de
		\begin{itemize}
			\item une origine,
			\item une coordonnée radiale $r$,
			\item une coordonnée angulaire $\alpha$.
		\end{itemize}
\end{itemize}
\subsection[Utilitaire]{utilities.hpp}
Le namespace ``utilities'' mis en place ici est un ensemble de fonctions et valeurs
constantes spécifiquement définies pour les calculs intervenant dans le projet.

\subsubsection{constantes :}
\begin{description}
	\item[PI] est un approximation de $\pi$ sur 26 décimales,
	\item[PI\_2] est une approximation de $\frac{\pi}{2}$ sur 26 décimales,
	\item[PI\_4] est une approximation de $\frac{\pi}{4}$ sur 26 décimales,
	\item[EPSILON] est une marge d'erreur de $10^{-7}$,
	\item[INF] représente un nombre dit ``infini'' dans le milieu informatique.
\end{description}
\subsubsection{fonctions :}
\begin{lstlisting}[title=Resolution d'équation du second degre,frame=single,language=C++]
utilitaire::secondDegreeEquationSolver
\end{lstlisting}
\begin{lstlisting}[title=Transforme un angle exprime en radian en un angle
exprime en degres,frame=single,language=C++]
utilitaire::angleAsDegree
\end{lstlisting}
\begin{lstlisting}[title=Permet de tester l egalite ou l inegalite entre deux 
nombre reels a un Epsilon d erreur,frame=single,language=C++]
utilitaire::equals
utilitaire::greaterOrEquals
utilitaire::lessOrEquals
\end{lstlisting}
\begin{lstlisting}[title=Permet de trouver le coefficiant angulaire a partir de deux points'',frame=single,language=C++]
utilitaire::slopeFromPoints
\end{lstlisting}
\begin{lstlisting}[title=Permet de trouver la valeur tangante d'un angle 
	en radian mais aussi de retourner une valeur particuliere pour la tangante de
$\pi/2$,frame=single,language=C++]
utilitaire::tan
\end{lstlisting}
\begin{lstlisting}[title=Permet de savoir si $\alpha$ vaut  $\frac{\pi}{2} + n *
\pi\ n \in \mathbb{N}$,frame=single,language=C++]
utilitaire::isHalfPiPlusNPi
\end{lstlisting}
\section[Les éléments]{elements}
\subsection[Element]{element.hpp}
Cette classe, abstraite et donc non instanciable, représente un element du jeu
lié à un, et un seul, niveau du jeu. Cette classe permet donc d'établir une
communication entre les différents éléments du niveau et le niveau lui-même à
travers un référencement de ce dernier.
Par le phénomène d'héritage, chaque élément poura établir sa manière propre de
réagir avec le niveau en lui exprimant : 
\begin{itemize}
	\item ses points d'intersection avec un rayon si il lui sont demandés,
		\begin{lstlisting}[language=C++]
		Element::includeRay(Ray);
		\end{lstlisting}
	\item les actions à éffectuer si un contact avec le rayon a eu lieu.
		\begin{lstlisting}[language=C++]
		Element::reactToRay(Ray);
		\end{lstlisting}
\end{itemize}
\subsection[Cristal]{crystal.hpp}
\subsection[Destination]{dest.hpp}
\subsection[Lentille]{lens.hpp}
\subsection[Niveau]{level.hpp}
\subsection[Createur de niveau]{levelfactory.hpp}
\subsection[Mirroir]{mirror.hpp}
\subsection[Bombe]{nuke.hpp}
\subsection[Rayon]{ray.hpp}
\subsection[Source]{source.hpp}
\subsection[Mur]{wall.hpp}
\section[L'exception]{exception}
\subsection[Exception Starlight]{starlightexception.hpp}
Il est nécessaire, pour bon nombre des classes créées, de valider les arguments passés en
paramètre dans le but de ne pas produire d'objets incohérents par rapport à
l'analyse préalable du travail à fournir. Pour ce faire, des exceptions doivent
être levées quand une instanciation créera un objet non désiré. 
Cette classe hérite de std::exception appartenant à la librairie standard. Elle n'a aucune
capacité supplémentaire mise à part être spécifique à ce projet. \\ \\

Les différentes classes pouvant lever cette exception sont :
\begin{description}
	\item[crystal] si la taille de son rayon ne lui permet pas d'exister dans le
		plan,
	\item[lens] si son interval de longueur d'onde n'est pas cohérent,
	\item[level] si ses dimensions ne lui permettent pas d'exister dans le plan,
	\item[mirror] si ses dimensions ne lui permettent pas d'exister dans le
		plan, si sa position ou son angle n'entre pas dans les limites imposées,
	\item[nuke] si la taille de son rayon ne lui permet pas d'exister dans le
		plan,
	\item[ray] si sa longueur d'onde n'entre pas dans l'interval cohérent
		imposé,
	\item[source] si sa longueur d'onde n'entre pas dans l'interval cohérent
		imposé,
	\item[wall] si ses points déterminants ne lui permettent pas d'exister dans
		le plan,
	\item[ellipse] si ses dimensions ne lui permettent pas d'exister dans le
		plan,
	\item[rectangle] si ses dimensions ne lui permettent pas d'exister dans le
		plan.
\end{description}
\section[Les objets visuels]{view}


\chapter[Structure du programme]{Structure générale du programme}



\chapter[Algorithmes]{Détail des algorithmes utilisés}
\section[Réflexion]{Algorithme de réflection}

\section[Intersection]{Algorithme d'intersection}
\subsection[Deux droites]{Intersection de deux droites}
\subsection[Droite et rectangle]{Intersection d'une droite et d'un rectangle}
\subsection[Droite et ellipse]{Intersection d'une droite et d'une ellipse}



\chapter{Test effectués}



\chapter{Conclusion}


\appendix


\chapter{Réferences}


\end{document}
