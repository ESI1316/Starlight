\documentclass[a4paper,11pt]{report}

\usepackage[utf8]{inputenc}
\usepackage[french]{babel}
\usepackage[T1]{fontenc}
\usepackage[pdftex]{graphicx} 
\usepackage{url} 
\usepackage{listings}
\usepackage[bookmarks, colorlinks=false, pdfborder={0 0 0}, pdftitle={Starlight
: rapport}, pdfauthor={Kriwin Paul, Placentino Simon}, pdfsubject={Starlight},
pdfkeywords={C++, ISO/IEC 14882:2011, Starlight, Projet}]{hyperref} 

\newcommand{\HRule}{\rule{\linewidth}{0.5mm}}

\begin{document}
\begin{titlepage}
	\begin{center}

		\includegraphics[width=0.15\textwidth]{./logo}\\[1cm]

		\textsc{\LARGE H.E.B. Ecole Superieur d'Informatique}\\[1.5cm]

		\textsc{\Large Laboratoire de C++ : projet 2}\\[0.5cm]

		\HRule \\[0.4cm]
		{\huge \bfseries Starlight \\[0.4cm]}
		\HRule \\[1.5cm]

		\noindent
		\begin{minipage}[t]{0.4\textwidth}
			\begin{flushleft} \large
				\emph{Auteurs:}\\
				Paul \textsc{Kriwin}\\
				Simon \textsc{Placentino}
			\end{flushleft}
		\end{minipage}%
		\begin{minipage}[t]{0.4\textwidth}
			\begin{flushright} \large
				\emph{Titulaire du cours:} \\
				Dr.~Romain \textsc{Absil}
			\end{flushright}
		\end{minipage}

		\vfill

		{\large \today}

	\end{center}
\end{titlepage}

\tableofcontents

\chapter{Introduction}


\chapter[Les classes]{Présentation succinte des classes}
\section[Les objets géométriques]{geometry}
\subsection[Ellipse]{ellipse.hpp}
\subsection[Droite]{line.hpp}
\subsection[Rectangle]{rectangle.hpp}
\subsection[Point]{point.hpp}
\subsection[Utilitaire]{utilities.hpp}
\section[Les éléments]{elements}
\subsection[Element]{element.hpp}
\subsection[Cristal]{crystal.hpp}
\subsection[Destination]{dest.hpp}
\subsection[Lentille]{lens.hpp}
\subsection[Niveau]{level.hpp}
\subsection[Createur de niveau]{levelfactory.hpp}
\subsection[Mirroir]{mirror.hpp}
\subsection[Bombe]{nuke.hpp}
\subsection[Rayon]{ray.hpp}
\subsection[Source]{source.hpp}
\subsection[Mur]{wall.hpp}
\section[L'exception]{exception}
\subsection[Exception Starlight]{starlightexception.hpp}
Il faut, pour bon nombre des classes créées, valider les arguments passés en
paramètre dans le but de ne pas produire d'objets incohérents par rapport à
l'analyse préalable du travail à fournir. Pour ce faire, des exceptions doivent
être levées quand une instanciation créera un objet non désiré. 
Cette classe hérite de std::exception appartenant à la librairie standard. Elle n'a aucune
capacité supplémentaire mise à part être spécifique à ce projet.
\section[Les objets visuels]{view}

\chapter[Structure du programme]{Structure générale du programme}


\chapter[Algorithmes]{Détail des algorithmes utilisés}
\section[Réflexion]{Algorithme de réflection}
\section[Intersection]{Algorithme d'intersection}
\subsection[Deux droites]{Intersection de deux droites}
\subsection[Droite et rectangle]{Intersection d'une droite et d'un rectangle}
\subsection[Droite et ellipse]{Intersection d'une droite et d'une ellipse}


\chapter{Test effectués}


\chapter{Conclusion}


\appendix

\chapter{Réferences}


\end{document}
